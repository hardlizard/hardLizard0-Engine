%% bare_jrnl.tex
%% V1.4b
%% 2015/08/26
%% by Michael Shell
%% see http://www.michaelshell.org/
%% for current contact information.
%%
%% This is a skeleton file demonstrating the use of IEEEtran.cls
%% (requires IEEEtran.cls version 1.8b or later) with an IEEE
%% journal paper.
%%
%% Support sites:
%% http://www.michaelshell.org/tex/ieeetran/
%% http://www.ctan.org/pkg/ieeetran
%% and
%% http://www.ieee.org/

%%*************************************************************************
%% Legal Notice:
%% This code is offered as-is without any warranty either expressed or
%% implied; without even the implied warranty of MERCHANTABILITY or
%% FITNESS FOR A PARTICULAR PURPOSE! 
%% User assumes all risk.
%% In no event shall the IEEE or any contributor to this code be liable for
%% any damages or losses, including, but not limited to, incidental,
%% consequential, or any other damages, resulting from the use or misuse
%% of any information contained here.
%%
%% All comments are the opinions of their respective authors and are not
%% necessarily endorsed by the IEEE.
%%
%% This work is distributed under the LaTeX Project Public License (LPPL)
%% ( http://www.latex-project.org/ ) version 1.3, and may be freely used,
%% distributed and modified. A copy of the LPPL, version 1.3, is included
%% in the base LaTeX documentation of all distributions of LaTeX released
%% 2003/12/01 or later.
%% Retain all contribution notices and credits.
%% ** Modified files should be clearly indicated as such, including  **
%% ** renaming them and changing author support contact information. **
%%*************************************************************************


% *** Authors should verify (and, if needed, correct) their LaTeX system  ***
% *** with the testflow diagnostic prior to trusting their LaTeX platform ***
% *** with production work. The IEEE's font choices and paper sizes can   ***
% *** trigger bugs that do not appear when using other class files.       ***                          ***
% The testflow support page is at:
% http://www.michaelshell.org/tex/testflow/



\documentclass[journal]{IEEEtran}
%
% If IEEEtran.cls has not been installed into the LaTeX system files,
% manually specify the path to it like:
% \documentclass[journal]{../sty/IEEEtran}





% Some very useful LaTeX packages include:
% (uncomment the ones you want to load)


% *** MISC UTILITY PACKAGES ***
%
%\usepackage{ifpdf}
% Heiko Oberdiek's ifpdf.sty is very useful if you need conditional
% compilation based on whether the output is pdf or dvi.
% usage:
% \ifpdf
%   % pdf code
% \else
%   % dvi code
% \fi
% The latest version of ifpdf.sty can be obtained from:
% http://www.ctan.org/pkg/ifpdf
% Also, note that IEEEtran.cls V1.7 and later provides a builtin
% \ifCLASSINFOpdf conditional that works the same way.
% When switching from latex to pdflatex and vice-versa, the compiler may
% have to be run twice to clear warning/error messages.






% *** CITATION PACKAGES ***
%
%\usepackage{cite}
% cite.sty was written by Donald Arseneau
% V1.6 and later of IEEEtran pre-defines the format of the cite.sty package
% \cite{} output to follow that of the IEEE. Loading the cite package will
% result in citation numbers being automatically sorted and properly
% "compressed/ranged". e.g., [1], [9], [2], [7], [5], [6] without using
% cite.sty will become [1], [2], [5]--[7], [9] using cite.sty. cite.sty's
% \cite will automatically add leading space, if needed. Use cite.sty's
% noadjust option (cite.sty V3.8 and later) if you want to turn this off
% such as if a citation ever needs to be enclosed in parenthesis.
% cite.sty is already installed on most LaTeX systems. Be sure and use
% version 5.0 (2009-03-20) and later if using hyperref.sty.
% The latest version can be obtained at:
% http://www.ctan.org/pkg/cite
% The documentation is contained in the cite.sty file itself.






% *** GRAPHICS RELATED PACKAGES ***
%
\ifCLASSINFOpdf
  \usepackage[pdftex]{graphicx}
  % declare the path(s) where your graphic files are
  % \graphicspath{{../pdf/}{../jpeg/}}
  % and their extensions so you won't have to specify these with
  % every instance of \includegraphics
  % \DeclareGraphicsExtensions{.pdf,.jpeg,.png}
\else
  % or other class option (dvipsone, dvipdf, if not using dvips). graphicx
  % will default to the driver specified in the system graphics.cfg if no
  % driver is specified.
  % \usepackage[dvips]{graphicx}
  % declare the path(s) where your graphic files are
  % \graphicspath{{../eps/}}
  % and their extensions so you won't have to specify these with
  % every instance of \includegraphics
  % \DeclareGraphicsExtensions{.eps}
\fi


% correct bad hyphenation here
\hyphenation{op-tical net-works semi-conduc-tor}


\begin{document}
%
% paper title
% Titles are generally capitalized except for words such as a, an, and, as,
% at, but, by, for, in, nor, of, on, or, the, to and up, which are usually
% not capitalized unless they are the first or last word of the title.
% Linebreaks \\ can be used within to get better formatting as desired.
% Do not put math or special symbols in the title.
\title{hardLizard0 Engine}
\author{John Emory,\\hardLizard Studios}
\maketitle

\begin{abstract}
The hardLizard0 Engine is a Game Engine written in golang targeting lower end hardware with tile based games.
The engine uses ebiten as a rendering backend.
The engine implements an Entity-Component-Systems (ECS) framework to achieve better performance on lower end hardware.
\end{abstract}




\section{Introduction}

\IEEEPARstart{T}{he} hardLizard0 Engine uses an ECS framework to provide a framework for tile based games on lower end hardware.
The Entity-Component-Systems (ECS) framework allows the engine to prioritize cache coherency in the CPU, avoiding cache misses. 
The use of tiles and spritesheets for graphics minimizes graphics memory used while capturing a retro aesthetic.
The rendering engine Ebiten is used to handle video output, player input, and audio.
For the hardLizard0 engine, minimal custom tooling for designers is to be developed in order to prioritize engine development. 
Tiled is the preferred level design tool.

\hfill July 13, 2019

\section{Design}
Explain what you are designing because at this point the reader doesn't know what you're talking about yet. Explain what you're making and how you intend to achieve it. Under every section heading you MUST say what subsections are contained within it, think of it as a baby introduction. Write the subheadings first tho.

An ECS framework is an alternative approach to the usual object oriented programing (OOP) paradigm often used for game engines.
Instead of creating and destroying objects that may be fragmented in memory, an ECS framework consists of Entities, Components and Systems.
Entities are restricted to unique Global Identifiers (GIDs) and hold no state of their own. 
Instead, a GID refers to the index of any of several arrays, called Components.
A Component is an array that holds information that is common between entities.
A System is a function that iterates over one or several Components, updating the global state that the Components constitute.

\subsection{Inspirations}
Drawing from the 16-bit aesthetic of the SNES, hardLizard0 aims to recapture this retro-aesthetic while simultaneously lifting some of the restrictions imposed by the state of the art in the 1990s.
Even the most basic hardware still in use today is orders of magnitude more powerful than the early 1990s video game consoles.
By lifting these restrictions, we can allow greater artistic breadth for game designers while retaining some of the architectural decisions that allowed that generation of games to be both performative and visually appealing.

\subsection{Goals}
The aim of the hardLizard0 engine is to provide a platform on which to build tile based games for modern, if lower powered, hardware.

\subsection{Software Design}
The use of an ECS framework ensures the engine will be maximally performative on minimal hardware.
A pitfall of OOP, though it may not be important in every program, is memory fragmentation.
When creating and destroying objects in memory, one does not have memory cohesion between common fields between objects.
Although an ECS framework increases design complexity of the engine, the potential performance gains, allowing less substantial hardware to run the engine, more than offsets the cost of complexity of design.



\section{Mechanics}
The hardLizard0 Engine aims to be a versatile engine that allows a game designer to realize a vision of a tile based game, irrespective of genera.
Included target generas are:
\begin{itemize}
  \item Platformer
  \item Metroidvania
  \item ActionAdventure
  \item Shoot'em-up
  \item Puzzle
\end{itemize}

The engine will be flexible enough to accommodate toggling gravity and different layer collisions.
A companion program for compiling json levels and png tilesheets and spritesheets into go files to be compiled into the final binary.
This will simplify deployment of binaries for various platforms.

\subsection{Level design}
For the preliminary version of the hardLizard Engine, the Tiled **insert link** map editor will be the prefered level editor.
The level compiler will accept world JSON files created by Tiled, supporting the following feature:
\begin{itemize}
  \item maps
  \item layers
  \item layer offsets
  \item tilesheets
  \item spritesheets
  \item animated tiles
  \item spawn points
  \item hitboxes
  \item map transitions 
\end{itemize}

\section{Development}
Initial work on a Tiled JSON loader has been completed.
For better performance, a custom tool to compile JSON into a go file will be needed.
Many Components have already been implemented. An ongoing list of features is maintained at https://github.com/hardLizard/hardLizard0.

\section{Pipeline}
Explain how the subs from DEVELOPMENT will be integrated, not how you'll code them but how you'll tie them in and the order in which you'll do them.

\section{Asset Development}
A common set of menu borders and rendered text should be provided for game designers to use, unifying a system aesthetic.

\section{Analysis}
Future versions of the engine should include support for simple 3D-graphics. 
Moving variables out of global variables while still avoiding heap allocation, or chosing a different language for development would be good to simplify retooling the main engine to fit a given project.

\section{Conclusion}

You should also learn how to do references and use literature to substantiate decisions as well as literature that challenges your decisions.

DRAFT DRAFT DRAFT DRAFTDRAFT DRAFTDRAFT DRAFTDRAFT DRAFT
\end{document}